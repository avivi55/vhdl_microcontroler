\documentclass[circuitt]{efrei_report_card}

\usepackage{geometry}
\geometry{
    a4paper,
    left=15mm,
    right=15mm,
}


\author{Jacques Soghomonyan}
\title{Rapport Projet}
\subtitle{Microcontrôleur en VHDL}

\graphicspath{ {./images} }

\usepackage{capt-of}

\begin{document}
    \maketitle

    \note{Les images de simulation sont parfois assez petite et un zoom peut être nécessaire pour les comprendre.}

    \exercise[0]{Introduction}{
        Ce projet a pour but de nous introduire au monde des cartes programmables FPGA.\\
        Pour ce faire nous avons programmé, en VHDL, une architecture simple de microcontrôleur.
        Ici, nous allons détailler les aspects techniques de toutes les composantes.
    }

    \exercise{Plan d'attaque.}{
        \question{}{
            Nous devons diviser le problème pour qu'il convienne au modèle entité de VHDL.

            L'énoncé est exhaustif sur la manière de faire et  les éléments qui composent le système.
            \medskip
            Dans le sujet, un schéma nous est donné. J'ai décidé de diverger un peu.
            
            En effet, certaines connexions semblaient redondantes p.~ex.~connexion de l'horloge à l'interconnexion.
            De plus, d'autres étaient implicite p.~ex.~les \textit{enable} des \textit{buffer}.
        }
    }




\end{document}